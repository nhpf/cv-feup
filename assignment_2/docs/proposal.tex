\documentclass[a4paper,12pt]{article}
\usepackage[left=10.0mm, right=10.0mm, top=25mm, bottom=25mm,footskip=20pt]{geometry}

\usepackage[utf8]{inputenc}

% Fonts
%---------------------------------------------------------
\usepackage{fontspec}
\usepackage{indentfirst}
\usepackage[colorlinks,urlcolor=blue]{hyperref}
\pagenumbering{gobble}

\usepackage{titling}
\setlength{\droptitle}{-3em}
\title{
\textbf{Multiple Vessel Detection and Tracking in Harsh Maritime Environments}
}
\author{
    Hopf, Nicholas\\
    % \textit{Faculty of Engineering}\\\textit{University of Porto}\\
    up200000000
    \and
    Gomes, Rodrigo\\
    % \textit{Faculty of Engineering}\\\textit{University of Porto}\\
    up200000000
    \and
    Colaço, Rui\\
    % \textit{Faculty of Engineering}\\\textit{University of Porto}\\
    up200000000
}
\date{\vspace{-3ex}}
% \date{}

% Fancy header
\usepackage{fancyhdr}

\fancypagestyle{plain}{\pagestyle{fancy}}
\fancyhf{}
\fancyhead[R]{Project proposal - Computer Vision}
\fancyhead[L]{April 2023}
% \fancyfoot[L]{}
% \fancyfoot[R]{Página \thepage}
%---------------------------------------------------------

\begin{document}

\maketitle

\section{Motivation}

This project will be based upon the work developed on paper \textit{Multiple Vessel Detection and Tracking in Harsh Maritime Environment} (Duarte D, Pereira M and Pinto A, 2023) as suggested by professors Andry Pinto and Daniel Campos as an introduction to the current computer vision research initiatives ongoing at FEUP and INESC TEC.
\\\par
The aforementioned paper will serve as the main guide to strategies and results that can be achieved regarding this project's scope, as well as a source of references that may support this project's development.

\section{Objectives}
As research concerning navigation of Autonomous Surface Vehicles (ASVs) increases, improvement in collision avoidance methods is critical. Especially in the maritime environment, where obstacles and the ASV may both be moving, a reliable tracking of surrounding vessels and motion prediction become even more relevant.
\\\par
Therefore, this project aims to detect and track vessels neighboring a reference camera that may be attached to an ASV. This context is especially relevant since real-time monitoring and adjustment of lighting and focus will not be performed, demanding robustness from the algorithms even with harsh environmental conditions.

\section{Problem statement}
Given that this project aims to develop a tracking model, transfer learning techniques will be used (based on YOLO-v4) to ensure object detection, while motion tracking will be implemented with the DeepSORT algorithm. The PyTorch framework will be used to all tasks regarding neural network training.

\section{Datasets}
The datasets that will be used within the scope of this project are:
\begin{itemize}
    \item the \href{https://sites.google.com/site/dilipprasad/home/singapore-maritime-dataset}{Singapore Maritime Dataset}, whose videos will be used to train and evaluate the object tracking tests. The videos are accompanied by matrix files of the bounding boxes of every object on frame.
    \item the \href{https://github.com/avaapm/marveldataset2016}{MARVEL Dataset}, that contains the images which will be used to train the classifier in the tracking module. It has 100k images of different vessels distributed over 38 classes.
\end{itemize}

\end{document}
